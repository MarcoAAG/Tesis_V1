%\setcounter{equation}{0}
\chapter{Introducci\'on}
 	Este capítulo cubre los antecedentes de esta tesis, así como sus propósitos, objetivos y limitaciones originales
\section{Preámbulo}
El incremento reciente del uso de cámaras digitales en sistemas UAVs para el uso de producciones profesionales cinematográficas ha hecho que diversos científicos e ingenieros estén interesados en diseñar controladores para estabilizar la posición de referencia de dichas cámaras con el objetivo de mitigar la mayor cantidad de ruido en las filmaciones.\\
El desarrollo de algoritmos genéticos ha favorecido en la implementación de controladores para los sistemas estabilizadores por lo que ha dado apertura a múltiples creaciones de algoritmos con la finalidad de obtener el mejor controlador con respecto al tiempo de respuesta. \\
Debido a la versatilidad de los UAVs, la implementación de una cámara no solo se limita al uso cinematográfico, va más allá, un ejemplo es la investigación realizada por Georgia Tech UAV quienes dirigen su trabajo hacia la navegación mediante un control de visión artificial.  

\section{Objetivos}
\subsection{Objetivo general}
Diseñar, simular y programar un controlador de estabilidad para un sistema gimbal de dos grados de libertad mediante el uso de visión artificial para detectar figuras geométricas en un marco de referencia inercial.
\subsection{Objetivos espec\'ificos}
\begin{itemize}
\item Obtener el modelo matemático de un estabilizador gimbal de dos grados de libertad, con base en la propuesta realizada por Maher Abdo.
\item Diseñar un controlador autónomo con base en el modelo matemático previamente obtenido, para estabilizar la superficie del sistema gimbal y con ello mitigar las perturbaciones de entrada. 
\end{itemize}

\section{Planteamiento y justificaci\'on}
El trabajo presentado en esta tesis tiene su justificación académica, porque durante las fases del modelado matemático, diseño e implementación del controlador y aplicación de técnicas de visión artificial engloba de manera concreta los conocimientos adquiridos relacionados al campo de los sistemas embebidos, tales como, electrónica de control, control en tiempo continuo y discreto y además se integra a estudios aerodinámicos, debido a que la gimbal fue colocada en un sistema UAV.







