
% this file is called up by thesis.tex
% content in this file will be fed into the main document
%----------------------- introduction file header -----------------------
%%%%%%%%%%%%%%%%%%%%%%%%%%%%%%%%%%%%%%%%%%%%%%%%%%%%%%%%%%%%%%%%%%%%%%%%%
%  Capítulo 1: Introducción- DEFINIR OBJETIVOS DE LA TESIS              %
%%%%%%%%%%%%%%%%%%%%%%%%%%%%%%%%%%%%%%%%%%%%%%%%%%%%%%%%%%%%%%%%%%%%%%%%%

\chapter{Introducción}

%: ----------------------- HELP: latex document organisation
% the commands below help you to subdivide and organise your thesis
%    \chapter{}       = level 1, top level
%    \section{}       = level 2
%    \subsection{}    = level 3
%    \subsubsection{} = level 4
%%%%%%%%%%%%%%%%%%%%%%%%%%%%%%%%%%%%%%%%%%%%%%%%%%%%%%%%%%%%%%%%%%%%%%%%%
%                           Presentación                                %
%%%%%%%%%%%%%%%%%%%%%%%%%%%%%%%%%%%%%%%%%%%%%%%%%%%%%%%%%%%%%%%%%%%%%%%%%

En el presente capitulo se expone el objetivo general, así como sus derivados. En la primera sección se aborda el tema de investigación donde especifica la justificación del presente trabajo, posteriormente se sintetiza algunas de las investigaciones que sirvieron como base para la elección del tema previamente descrito. Finalmente se dan las razones de la investigación y se exponen las aportaciones derivadas del tema de tesis.

\section{Tema de investigación}

\section{Justificación}

%%%%%%%%%%%%%%%%%%%%%%%%%%%%%%%%%%%%%%%%%%%%%%%%%%%%%%%%%%%%%%%%%%%%%%%%%
%Se escribe en infinitivo y resuelve las preguntas ¿Que? ¡Como? y ¿para que? %
%%%%%%%%%%%%%%%%%%%%%%%%%%%%%%%%%%%%%%%%%%%%%%%%%%%%%%%%%%%%%%%%%%%%%%%%%
En la actualidad el uso de cámaras digitales en los sistemas UAV’s cumplen una enorme función para la obtención de datos gráficos que permitan la realización de algoritmos capaces de generar múltiples tareas tales como, navegación inercial, búsqueda y rescate, entre otros. Debido a la criticidad que puede llegar a tornarse este tipo de tareas donde la comunicación entre la tarjeta que captura datos de la cámara y un sistema receptor necesitan protocolos seguros de comunicación , por tal razón este proyecto tiene la finalidad de crear un sistema embebido que pueda ser autónomo y se encuentre instalado en un sistema UAV, con esto beneficiar la integridad del producto y generando un entorno seguro para el procesamiento de datos de la cámara hacia la gimbal.
\section{Objetivo}

Diseñar, instrumentar y controlar un dispositivo gimbal que sea capaz de seguir un objeto a través de visión artificial para implementarse en un UAV de categoría pequeña.

%%%%%%%%%%%%%%%%%%%%%%%%%%%%%%%%%%%%%%%%%%%%%%%%%%%%%%%%%%%%%%%%%%%%%%%%%
%Seran los capitulos de la tesis %
%%%%%%%%%%%%%%%%%%%%%%%%%%%%%%%%%%%%%%%%%%%%%%%%%%%%%%%%%%%%%%%%%%%%%%%%%
\section{Objetivos específicos}
\begin{itemize}
\item Obtener el modelo matemático de una gimbal de 2 grados de libertad.
\item Diseñar e implementar el sistema embebido que dará el soporte electrónico a la gimbal.
\item Capturar figuras geométricas definidas  mediante el uso de una cámara digital y emplear algoritmos de visión artificial para la obtención de datos. 
\item Diseñar un controlador autónomo con base en el modelo matemático, previamente obtenido.
\end{itemize}


%%%%%%%%%%%%%%%%%%%%%%%%%%%%%%%%%%%%%%%%%%%%%%%%%%%%%%%%%%%%%%%%%%%%%%%%%
%Descripción breve de las obras, proyectos, intentos universitarios más significativos %
%%%%%%%%%%%%%%%%%%%%%%%%%%%%%%%%%%%%%%%%%%%%%%%%%%%%%%%%%%%%%%%%%%%%%%%%%
\section{Estado de la cuestión}




%%%%%%%%%%%%%%%%%%%%%%%%%%%%%%%%%%%%%%%%%%%%%%%%%%%%%%%%%%%%%%%%%%%%%%%%%
%                           Motivación y estado del arte                %
%%%%%%%%%%%%%%%%%%%%%%%%%%%%%%%%%%%%%%%%%%%%%%%%%%%%%%%%%%%%%%%%%%%%%%%%%
\section{Marco teorico}


\blindtext

%%%%%%%%%%%%%%%%%%%%%%%%%%%%%%%%%%%%%%%%%%%%%%%%%%%%%%%%%%%%%%%%%%%%%%%%%
%                         Contribuciones                                %
%%%%%%%%%%%%%%%%%%%%%%%%%%%%%%%%%%%%%%%%%%%%%%%%%%%%%%%%%%%%%%%%%%%%%%%%%

\section{Contribuciones}

La principal contribución de este trabajo es 
\blindtext



%%%%%%%%%%%%%%%%%%%%%%%%%%%%%%%%%%%%%%%%%%%%%%%%%%%%%%%%%%%%%%%%%%%%%%%%%
%¿Porque es relevante la solucion o mejora o implementacion.? ¿por que es novedosa? %
%%%%%%%%%%%%%%%%%%%%%%%%%%%%%%%%%%%%%%%%%%%%%%%%%%%%%%%%%%%%%%%%%%%%%%%%%
\section{Alcances}
%%%%%%%%%%%%%%%%%%%%%%%%%%%%%%%%%%%%%%%%%%%%%%%%%%%%%%%%%%%%%%%%%%%%%%%%%
%                           Estructura de la tesis                      %
%%%%%%%%%%%%%%%%%%%%%%%%%%%%%%%%%%%%%%%%%%%%%%%%%%%%%%%%%%%%%%%%%%%%%%%%%

\section{Estructura de la tesis}

Este trabajo está dividido en XX capítulos. Al principio se encuentra 
\\\\
Finalmente se encuentra la parte de 