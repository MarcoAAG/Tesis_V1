\chapter{Introducción}
En el presente capitulo se expone el objetivo general, así como sus derivados. En la
primera sección se aborda el tema de investigación donde especifica la justificación del
presente trabajo, posteriormente se sintetiza algunas de las investigaciones que sirvieron
como base para la elección del tema previamente descrito. Finalmente se dan las razones
de la investigación y se exponen las aportaciones derivadas del tema de tesis.

\section{Tema de investigación}
En el campo de la aeronáutica hay una rama que en los últimos años ha sido objeto
de estudio debido a su exponencial importancia para tareas criticas, se trata de los
vehículos aéreos no tripulados UAV (del inglés unmanned aerial vehicle), donde dichas 
tareas criticas han podido alcanzar sus objetivos en parte gracias a la implementación
reciente de visión artificial, que dicho sea de paso ha dado pie a múltiples investigaciones
para generar una buena comunicación de datos entre el UAV y un sistema receptor en
tierra, dado que aveces las tareas requieren un tiempo de respuesta menor del que un
protocolo de comunicación puede otorgar o en donde se necesita garantizar la seguridad
tanto de software y hardware ha surgido la necesidad de diseñar un sistema embebido
con la finalidad de evitar los problemas relacionados con los protocolos de comunicación
y a su vez tener como resultado un sistema enteramente autónomo.

\section{Justificación}

\section{Objetivo}

\section{Objetivos específicos}

\section{Estado de la cuestion}

\section{Marco teorico}

\section{Contribuciones}

\section{Alcances}

\section{Estructura de la tesis}