\chapter{Marco Téorico}
\section{Óptica}
La visión artificial surge de un amplio estudio probabilístico y matemático del procesamiento de imágenes digitales, pero sobre todo de análisis humanos y de la intuición ya que de estas últimas el ingeniero hace selección de entre una u otra técnica. Esta elección se basa usualmente en juicios visuales subjetivos.\\
Entender los conceptos básicos de la percepción humana es entonces pertinente, donde la Óptica nos ayudará a entender mejor como es que el ojo humano percibe y como lo hace una cámara. \\
La función de la óptica de una cámara es captar los rayos luminosos y concentrarlos sobre el sensor sensible de la cámara de vídeo. Después de determinar el tipo de iluminación que mejor se adecua al problema, la elección de una óptica u otra influirá en la calidad de la imagen y el tamaño de los objetos.
\subsection{Estructura del ojo humano}
\begin{itemize}
\item Cornea: La córnea es una estructura del ojo que permite el paso de la luz desde el exterior al interior del ojo y protege el iris y el cristalino. Posee propiedades ópticas de refracción y para garantizar su función debe ser transparente y es necesario que mantenga una curvatura adecuada.
\item Esclerótica: Es el recubrimiento exterior blanco del ojo. La esclerótica le da su color blanco al globo ocular.
\item Coroides: Es la capa de vasos sanguíneos y tejido conectivo entre la parte blanca del ojo y la retina (en la parte posterior del ojo). Es parte de la úvea y suministra los nutrientes a las partes internas del ojo.
\item Cuerpo ciliar: Es una estructura circular que es una prolongación del iris, la parte de color del ojo. También contiene el músculo ciliar, el cual cambia la forma del cristalino cuando los ojos se enfocan en un objeto cercano. Este proceso se denomina acomodación.
\item Diafragma Iris: que se expande o contrae para controlar la cantidad de luz que entra en el ojo. La apertura central del iris, llamada pupila, varía su diámetro de 2 a 8mm. El frente del iris contiene el pigmento visible del ojo, y la parte trasera contiene un pigmento negro.
\item Cristalino: El cristalino es “la lente” del ojo y sirve para enfocar, ayudado por los músculos ciliares. El cristalino es una lente que actúa como una lente biconvexa, lenticular, flexible y avascular, cuya principal función es la de enfocar los objetos en las distintas distancias correctamente.
\item Retina: Es la capa de tejido sensible a la luz que se encuentra en la parte posterior globo ocular. Las imágenes que pasan a través del cristalino del ojo se enfocan en la retina. La retina convierte entonces estas imágenes en señales eléctricas y las envía por el nervio óptico al cerebro.
\end{itemize}
\subsection{Formación de imágenes en el ojo}
En una cámara fotográfica se recibe la luz que traspasa el diafragma, pasa por los cristales de la cámara hasta llegar al CCD(Charge Coupled Device o, en español, Dispositivo de Carga Acoplada) o sensor, que es donde se forma la imagen correcta y se envía al procesador.\\
Algo similar pasa en el ojo, la pupila es el diagrama natural que filtra la luz que entra en el ojo, pasa por la lente (el cristalino)  que converge los rayos hasta llegar a la retina, que es la estructura que tiene las células fotosensibles y dónde se produce la imagen, y a través del nervio óptico se transporta la información al cuerpo geniculado, que es la parte del cerebro donde se produce la visión.\\
El ojo esta formado de dos componentes principales:
\begin{itemize}
\item Componentes ópticos: permiten la formación de la imagen en la retina y son los siguientes: la córnea, el cristalino, la pupila, el humor acuoso y el humor vítreo que permiten la formación de una imagen en la retina.
\item Componentes neurológicos: son los que transforman la información óptica en eléctrica y transmiten la información al cuerpo geniculado lateral. Estos componentes son la retina y el nervio óptico.
\end{itemize}



